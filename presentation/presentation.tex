\documentclass{beamer}
\usepackage{listings}
\usepackage{fancyvrb}
\mode<presentation> {
\setbeamercovered{transparent}
\usetheme{PaloAlto}

}

\usepackage{graphicx}
\usepackage{booktabs} 

%----------------------------------------------------------------------------------------
%	TITLE PAGE
%----------------------------------------------------------------------------------------
\title[CoAP P/S]{Implementation of CoAP Publish/Subscribe using Python and Contiki-Os} 

\author[G. Carignani\\V. Consales\\F. Rossi]{Gioele Carignani, Vincenzo Consales, Federico Rossi} 
\institute[UNIPI] 
{
University of Pisa \\ 
}
\date{\today}

\begin{document}

\begin{frame}
\titlepage 
\end{frame}

\section{Overview}
\begin{frame}
\frametitle{Overview}
\tableofcontents 
\end{frame}

%----------------------------------------------------------------------------------------
%	PRESENTATION SLIDES
%----------------------------------------------------------------------------------------

\section{Introduction}

\subsection{Abstract} 
\begin{frame}
\frametitle{Abstract}
The work proposed is an extension of the CoAP protocol to enable publish-subscriber communication paradigm, employing a Broker node that enables store-and-forward messaging between constrained
nodes that can either publish or subscribe to particular objects.
\end{frame}

\subsection{Concepts}
\begin{frame}
\frametitle{Concepts}
\begin{block}{Topic}
A Topic is an item constrained nodes may subscribe or publish to. Its information is represented by its internal state.
\end{block}
\pause
\begin{block}{Broker}
A Broker node is a resource-rich component that maintains state of Topics and handles the publish/subscribe paradigm on the p/s nodes behalf.
\end{block}
\end{frame}

\begin{frame}
\frametitle{Concepts (cont.)}
\begin{block}{Publisher}
A Publisher is a constrained node that can update the state of a Topic by issuing requests to Broker.
\end{block}
\pause
\begin{block}{Subscriber}
A Subscriber is a constrained node that attach itself to a Topic, listening for changes from the Broker.
\end{block}
\end{frame}

\section{Implementation}

\subsection{Broker}
\begin{frame}
\frametitle{Broker}
Broker module is a Python server built over the \texttt{CoAPthon} library, by extending the base \texttt{CoAP} server class adding required functionalities.
\end{frame}

\subsubsection{PS Resource}
\begin{frame}
\frametitle{PS Resource}
The PS APIs have been implemented by means of a \textit{CoAP Resource} that, according to the draft RFC \textit{draft-ietf-core-coap-pubsub-05}, is exposed with the \textit{Uri-Path} \texttt{/ps}.
\pause
The resource allows the four CoAP methods:
\begin{itemize}
	\item GET: implements the \texttt{READ} and, with CoAP \texttt{observe:0}, the \texttt{SUBSCRIBE} functionalities.\pause

	\item POST: implements the \texttt{CREATE} functionality.\pause

	\item PUT: implements the \texttt{PUBLISH} and \texttt{CREATE on PUBLISH} functionalities.\pause

	\item DELETE: implements the \texttt{DELETE} functionality.
\end{itemize}
\end{frame}

\begin{frame}
\frametitle{PS Resource (cont.)}
A modification to the original \texttt{CoAPthon} library is necessary to handle \texttt{CREATE on PUBLISH} behavior described in the draft RFC \textit{draft-ietf-core-coap-pubsub-05}: \pause
\begin{itemize}
	\item Need to intercept PUT request directed to a non-existent topic in order to create it and then answer to the request. \pause
	\item Otherwise the default behavior would be a \texttt{4.04 Not Found} response.
\end{itemize}
\pause
The other functionalities are already well implemented by the \texttt{CoAPthon} library so they can be used as they are.
\end{frame}


\subsubsection{Plus: Docker container}
\begin{frame}
\frametitle{Plus: Broker Docker container}
The Docker container has been defined in order to ease deployment of the Broker application and IPv6 configuration of the server machine.
\end{frame}

\begin{frame}[fragile]
\frametitle{Plus: Broker Docker container (cont.)}
\begin{exampleblock}{Dockerfile}
	\begin{Verbatim}[fontsize=\footnotesize]
FROM python:2.7
ADD *.py /
RUN git clone https://github.com/federicorossifr/CoAPthon.git
RUN ls CoAPthon
RUN cd CoAPthon && python setup.py sdist
RUN pip install CoAPthon/dist/CoAPthon-4.0.2.tar.gz
EXPOSE 5683
CMD [ "python", "app.py" ]
	\end{Verbatim}
\end{exampleblock}
\end{frame}

\begin{frame}[fragile]
\frametitle{Plus: Broker Docker container (cont.)}
\begin{exampleblock}{docker-compose.yaml}
	\begin{Verbatim}[fontsize=\footnotesize]
version: '2.1'
services:
  app:
    build: .
    ports:
        - "5683:5683"
    networks:
      app_net:
        ipv6_address: "2402:9400:1000:7::FFFF"
networks:
  app_net:
    enable_ipv6: true
    ipam:
      config:
        - subnet: "2402:9400:1000:7::/64"
	\end{Verbatim}
\end{exampleblock}
\end{frame}

\subsection{Pub/Sub nodes}
\frametitle{Publisher/Subscriber nodes}
\begin{frame}
	Publisher and subscriber nodes are implemented using the \texttt{Contiki} platform and the \texttt{Erbium} CoAP framework
\end{frame}

%------------------------------------------------
\section{Use-Case scenario}
\begin{frame}
\frametitle{Anti-intrusion system} 
\begin{itemize}
  \item Publisher node (Zolertia Z1)\pause
  \begin{itemize}
    \item Creates the topic \textit{/alarm} on the broker\pause
    \item When activated (i.e. button pressed),starts sensing forces on the door using the accelerometer\pause
    \item Publishes an alarm value on the topic when a certain force is detected\pause
    \item Another button press disable the alarming functionality and reset the topic value (for DEMO purposes only. Typically a numeric code is required to disable it)
  \end{itemize}\pause

  \item Subscriber node (Zolertia Z1)\pause
  \begin{itemize}
    \item Subscribes to the \textit{/alarm} topic on the broker\pause
    \item When it is notified about an alarm value it lights on its led (for demo purposes only. An enough powerful tweeter would be better to notify the intrusion)
  \end{itemize}
\end{itemize}
\end{frame}


%------------------------------------------------
\section{References}
\begin{frame}
\frametitle{References}
\footnotesize{
\begin{thebibliography}{99} % Beamer does not support BibTeX so references must be inserted manually as below
\bibitem[G. Tanganelli,C. Vallati, E. Mingozzi 2016]{p1}G. Tanganelli,C. Vallati,E. Mingozzi (2016)
\newblock CoAPthon: Easy development of CoAP-based IoT applications with Python
\newblock \emph{2015 IEEE 2nd World Forum on Internet of Things (WF-IoT)}
\bibitem[M. Koster,A. Keranen,J. Jimenez 2018]{p1}G. M. Koster,A. Keranen,J. Jimenez (2018)
\newblock Publish-Subscribe Broker for the Constrained Application Protocol (CoAP)

\newblock \emph{draft-ietf-core-coap-pubsub-05}
\end{thebibliography}
}
\end{frame}
%------------------------------------------------
\section{Demo}
\begin{frame}
\Huge{\centerline{Demo Time!}}
\end{frame}


%------------------------------------------------
\section{EOF}
\begin{frame}
\Huge{\centerline{EOF}\centerline{Questions?}}
\end{frame}

%----------------------------------------------------------------------------------------

\end{document} 